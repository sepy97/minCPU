\documentclass{article}
\usepackage[utf8]{inputenc}
\usepackage[russian]{babel}
\begin{document}

\section{Задача оптимизации скорости процесса при заданном расписании задач}

$\min W$, где $W = \sum \limits_{i=0}^{T-1} \alpha + \beta x_i + \gamma x_i^2$, $x_i$ - скорость процессора в момент времени i\\

s.t. $S_{min} \le x_i \le S_{max}$\\

$|x_i-x_{i-1}| \le R ~~ \forall i = 1 \dots T-1$, где T - суммарное время работы процессора \\

Модуль можно записать как два ограничения: \\

$x_i-x_{i-1} \le R ~~ \forall i = 1 \dots T-1$\\
$-x_i+x_{i-1} \le R ~~ \forall i = 1 \dots T-1$\\

$r_{ij}$ - ресурсы, расходуемые на i-м промежутке времени на j-м процессе\\

$x_i = \sum \limits_{j=0}^{n-1} r_{ij}\delta_{ij}$, $n$ - число задач, $\delta_{ij}$ - элементы матрицы доступности процессов (доступен ли j-ый процесс в i-ый момент времени)\\

$\sum \limits_{i=0}^{T-1} r_{ij}\delta_{ij} \ge z_j$, где $z_j$ - работа, которую надо затратить для j-го процесса\\
\\
\\


\underline{Формально:}\\
Вычислить до постановки задачи: \\
$$x_i = \sum\limits_j N_{ij}$$
$$y_j = \sum\limits_i N_{ij}$$
$$x^* = (x_1, \dots, x_{15})$$
$$x_* = (x_0, \dots, x_{14})$$

Получаем задачу:\\
$\min \alpha T + \mathbf{\beta^{\top}x} + \gamma \mathbf{x^{\top}x}$\\
s.t. $\mathbf{x}-\mathbf{S_{min}} \succeq 0$\\
$\mathbf{S_{max}}-\mathbf{x} \succeq 0$\\
$\mathbf{x^*} - \mathbf{x_*} \preceq \mathbf{R}$\\
$\mathbf{x_*} - \mathbf{x^*} \preceq \mathbf{R}$\\
$\mathbf{y} \succeq \mathbf{z}$\\

Решаем барьерным методом (логарифмический барьер), 
полученную задачу без ограничений засовываем в градиентный метод.


\end{document}
